\documentclass[]{article}

\usepackage[left=2.00cm, right=2.00cm, top=2.00cm, bottom=2.00cm]{geometry}
\usepackage[spanish,es-noshorthands]{babel}
\usepackage[utf8]{inputenc} % para tildes y ñ
\usepackage{graphicx} % para las figuras
\usepackage{xcolor}
\usepackage{listings} % para el código fuente en c++

\lstdefinestyle{customc}{
  belowcaptionskip=1\baselineskip,
  breaklines=true,
  frame=single,
  xleftmargin=\parindent,
  language=C++,
  showstringspaces=false,
  basicstyle=\footnotesize\ttfamily,
  keywordstyle=\bfseries\color{green!40!black},
  commentstyle=\itshape\color{gray!40!gray},
  identifierstyle=\color{black},
  stringstyle=\color{orange},
}
\lstset{style=customc}


%opening
\title{Práctica 1. Algoritmos devoradores}
\author{Francisco Javier Molina Rojas \\ % mantenga las dos barras al final de la línea y este comentario
javier.molinarojas@alum.uca.es \\ % mantenga las dos barras al final de la linea y este comentario
Teléfono: 722528757 \\ % mantenga las dos barras al final de la línea y este comentario
NIF: 45386606Q \\ % mantenga las dos barras al final de la línea y este comentario
}


\begin{document}

\maketitle

%\begin{abstract}
%\end{abstract}

% Ejemplo de ecuación a trozos
%
%$f(i,j)=\left\{ 
%  \begin{array}{lcr}
%      i + j & si & i < j \\ % caso 1
%      i + 7 & si & i = 1 \\ % caso 2
%      2 & si & i \geq j     % caso 3
%  \end{array}
%\right.$

\begin{enumerate}
\item Describa a continuación la función diseñada para otorgar un determinado valor a cada una de las celdas del terreno de batalla para el caso del centro de extracción de minerales. 

$$ f(salud,ataqueporseg,danio,dispersion,coste)= \frac{{salud}+{ataqueporseg}+{danio}+{dispersion}}{coste} $$

Para realizar la funcion que dara un valor a cada defensa, deberemos tener en cuenta los atributos de esta, intentando sacar los maximos de estos en relación a su coste.
Los atributos que dan valor a la defensa son: los ataques por segundo (attacksPerSecond), el daño (damage), la dispersion (dispersion) y la vida (health).
El atributo que resta valor a la defensa es su coste (cost).
Entonces haremos un promedio devolviendo la suma de los atributos que dan valor a la defensa entre su coste.

\begin{lstlisting}[frame=single,basicstyle=\tiny,title={Funcion DefenseValue}]
float DefenseValue(std::list<Defense*>::iterator currentDef)
{
	return ((*currentDef)->health + (*currentDef)->attacksPerSecond + 
	(*currentDef)->damage + (*currentDef)->dispersion)/(*currentDef)->cost;
}
\end{lstlisting}


\item Diseñe una función de factibilidad explicita y descríbala a continuación.

La función de factibilidad comprobará si una celda es factible para colocar la defensa o no. Para ello, tendremos que comprobar que no este fuera de rango, que no choque con un obstáculo o defensa ya colocada y que no salga del mapa. \\\\
Para que la casilla no este fuera de rango, no tiene que ser menor que 0 tanto en fila como en columna y no tiene que ser mayor o igual que los extremos del tablero.\\\\
Para que no choque con los obstáculos o defensas, la distancia entre el centro de la celda candidata y la celda del obstáculo/defensa ya colocada no debe ser menor que la suma de los radios de la defensa a colocar y el obstáculo/defensa ya colocada.\\\\
Para que no salga la defensa del mapa tenemos que comprobar que su radio no viola los limites del mapa. Para ello si la posición de la fila o la posición de la columna menos el radio de la defensa es menor que 0 sobrepasa alguno de los extremos inferiores. Mientras que si la posición de la fila o la posición de la columna mas el radio de la defensa es mayor que la anchura o la altura del mapa, entonces sobrepasa alguno de los extremos superiores.


\begin{lstlisting}[frame=single,basicstyle=\tiny,title={Función de factibilidad}]
bool EsFactible(int row, int col,List<Defense*>::iterator currentdefense
	, int nCellsWidth, int nCellsHeight, float mapWidth, float mapHeight, 
	List<Object*> obstacles, List<Defense*> defenses,int numDefensasColocadas)
{
	//definimos una variable factible, la predefinimos con true, 
	//y si encontramos que no es factible la pondremos a false
	bool factible = true; 
	
	//si la row y col de la celda pasada esta dentro de rango
	if(row >= 0 && row < nCellsWidth && col >= 0 && col < nCellsHeight) 
    {
		//definicon de variables necesarias para la verificacion de la factibilidad
		//obtenemos la posicion de la celda
		Vector3 pos = cellCenterToPosition(row,col,mapWidth/nCellsWidth
		,mapHeight/nCellsHeight);  
		//definimos un iterador para la lista de obstulos inicializado con la primera posicion
		List<Object*>::iterator poit = obstacles.begin(); 
		//definimos un iterador para la lista de defensas inicializado con la primera posicion
		List<Defense*>::iterator pdit = defenses.begin(); 

		//Comprobamos que los obstaculos no chocan con nuestra defensa
		//mientras que haya obstaculos por comprobar y factible sea verdadero
		while(poit != obstacles.end() && factible)	
		{
			//si la suma de los radios de la defensa a colocar y el radio del obstaculo es mayor que 
			//la distancia desde el centro de la posicion
			//de la celda hasta el centro de la posicion del obstaculo
			if(((*currentdefense)->radio + (*poit)->radio) > 
			_distance(pos,(*poit)->position)) 
				factible = false; //entonces la defensa no es colocable ya que estaria 
				//intersectando el radio del ocbstaculo (factible es falso)
			poit++; // pasamos al siguiente obstaculo
		}
		
		//Comprobamos que las defensas YA COLOCADAS no chocan con nuestra defensa
		//mientras que haya defensas por comprobar y factible sea verdadero
		while(pdit != defenses.end() && factible) 
		{
			//si la defensa que iteramos es la que vamos a colocar o 
			//numero de defensas colocadas es 0, no hay defensa que pueda colisionar
			if(pdit == currentdefense || numDefensasColocadas == 0) 
			{
				pdit++; //pasamos a la siguiente defensa
				continue; //terminamos la iteracion actual en el bucle
			}
			else
			{
				//si la suma del radios de la defensa a colocar 
				//y el radio de la defensa ya colocada es mayor que 
				//la distancia desde el centro de la posicion de la
				//celda hasta el centro de la posicion defensa
				if( ((*currentdefense)->radio + (*pdit)->radio) 
				> _distance(pos,(*pdit)->position))
					factible = false; //entonces la defensa no 
					//es colocable ya que estaria
				//intersectando el radio de la otra defensa (factible es falso)
				pdit++; // pasamos a la siguiente defensa
				numDefensasColocadas--; //hemos comprobado esta defensa, 
				//por lo que ya no la contamos
			}
		}

		//si la resta de la posicion x/y de la celda 
		//menos el radio de la defensa es menor que 0
		//o la suma de la posicion x/y de la celda 
		//mas el radio de la defensa es mayor que
		//la anchura o altura del mapa entonces el radio de la defensa
		//ocupa una parte inexistente del borde del mapa
		if(pos.x - (*currentdefense)->radio < 0 
		|| pos.y - (*currentdefense)->radio < 0 
		||pos.x + (*currentdefense)->radio > mapWidth 
		|| pos.y + (*currentdefense)->radio > mapHeight)
			factible = false; //entonces la defensa no es colocable ya que 
			//estaria violando los bordes del mapa (factible es falso)
	}
	else	//Si row y col esta fuera de rango
		factible = false; //entonces la defensa no es colocable (factible es falso)

	return factible; //devolvemos la factible
}

\end{lstlisting}


\item A partir de las funciones definidas en los ejercicios anteriores diseñe un algoritmo voraz que resuelva el problema para el caso del centro de extracción de minerales. Incluya a continuación el código fuente relevante. 

\begin{lstlisting}[frame=single,basicstyle=\tiny,title={Voraz para el centro de extracción}]
void DEF_LIB_EXPORTED placeDefenses(bool** freeCells, int nCellsWidth, 
			int nCellsHeight, float mapWidth, float mapHeight
            , std::list<Object*> obstacles, std::list<Defense*> defenses) {
	
    float cellWidth = mapWidth / nCellsWidth; //Ancho de celda
    float cellHeight = mapHeight / nCellsHeight; //Altura de celda
    int numDefensasColocadas = 0; //Numero de defensas colocadas

	//Definicion de variables que usaremos mas adelante:
	//Para la lista de candidatos usaremos una lista la 
	//cual almacenara las celdas de manera ordenada conforme al valor de la celda
	std::list<Celda> Celdas;
	//Definimos una celda auxiliar para poder seleccionar un candidato de la lista
	Celda auxiliar(0,0,0);
	//definimos un iterador de la lista de defensas para poder colocar las defensas
	List<Defense*>::iterator currentDefense = defenses.begin();
		
	//Inicializamos la lista de candidatos para colocar el centro de extraccion
	for(int i = 0 ; i < nCellsWidth ; i++)
	{
		for(int j = 0 ; j < nCellsHeight ; j++)
		{
			//insertamos todas las celdas en la lista de candidatos. Para poder 
			//ordenarla luego usaremos el metodo sort
			Celdas.push_front(Celda(i,j,cellValueExtrationCenter(i,j
			,freeCells,cellWidth,cellHeight,nCellsWidth,nCellsHeight)));
		}
	}
	Celdas.sort(); //Se realiza la ordenacion (al final de 
	//las listas estan las mejores celdas)

	//Comenzaremos colocando el centro de extraccion de minerales 
	//(primera posicion de defensas)
	//Definimos una variable que nos indicara true cuando 
	//el centro de extraccion sea colocado
	bool CentroColocado = false; 
	// Mientras que hayan candidatos(celdas)disponibles 
	//y el centro no haya sido colocado
	while(!Celdas.empty() && !CentroColocado) 
	{
		auxiliar = Celdas.back(); //Guardamos el mejor candidato posible
		Celdas.pop_back(); //lo eliminamos de la lista de candidatos
		//Si la celda elegida es factible
		if(EsFactible(auxiliar.row,auxiliar.col,currentDefense,nCellsWidth,
		nCellsHeight,mapWidth,mapHeight,
		obstacles,defenses,numDefensasColocadas))  
		{
			//le damos a la primera defensa (Centro de Extraccion) 
			//la posicion de la celda elegida
			(*currentDefense)->position = 
			cellCenterToPosition(auxiliar.row,auxiliar.col,
			cellWidth,cellHeight); 
			//El centro ha sido colocado
			CentroColocado = true; 
			//numero de defensas es 1
			numDefensasColocadas++; 
		}
	}
\end{lstlisting}


\item Comente las características que lo identifican como perteneciente al esquema de los algoritmos voraces. 

\begin{lstlisting}[frame=single,basicstyle=\tiny,title={Algoritmo para rellenar la tabla de subproblemas resueltos}]
void DEF_LIB_EXPORTED selectDefenses(std::list<Defense*> defenses, unsigned int ases, std::list<int> &selectedIDs
            , float mapWidth, float mapHeight, std::list<Object*> obstacles)             
{
    //definicion de variables necesarias
    //creamos vector de candidatos
    std::vector<candidato> Candidatos; 
    //iterador para recorrer las defensas
    std::list<Defense*>::iterator defit = defenses.begin(); 
    
    //No contamos el centro de extraccion ya 
    //que este es necesario para el funcionamiento
    //metemos la primera defensa(centro de extraccion)
    selectedIDs.push_back((*defit)->id);
    //pasamos a la siguiente defensa
    defit++;


    //bucle para inicializar la lista de candidatos
    while(defit != defenses.end()) 
    {   
     	//Cada candidato tendra un puntero a la defensa y valor que tiene la misma defensa
        Candidatos.push_back(candidato((*defit),DefenseValue(defit)));
        defit++;//pasamos a la siguiente defensa
    }
    //ordenamos el vector (de menor a mayor coste)
    std::sort(Candidatos.begin(),Candidatos.end());
    

    //utilizaremos una matriz de flotantes para poder representar la tabla de subproblemas resuletos
    //ademas, esta debera ser de dimension 
    //numdecandidatos(defensas) * presupuesto + 1 (con el objetivo de poder llegar al valor de ases
    float evaluacionTotal[Candidatos.size()][ases+1]; //definicion de tabla de subproblemas

	//Bucle para recorrer la tabla de subproblemas
    for(int i = 0 ; i < Candidatos.size() ; i++)
    {
        for(int j = 0 ; j < ases+1 ; j++)
        {
        	//si estamos en la primera fila y j (presupuesto actual) es mayor o igual a 
        	//lo que me cuesta la defensa, entonces la meto en la "mochila"
            if(i == 0 && j >= Candidatos[i].def->cost)	
                evaluacionTotal[i][j] = Candidatos[i].valor;
                
            //si estamos en la primera fila y j (presupuesto actual) es menor a
            //lo que me cuesta la defensa, entonces no meto nada en la mochila
            if(i == 0 && j < Candidatos[i].def->cost)
                evaluacionTotal[i][j] = 0;
                
            //si estamos en una fila superior a la primera
            if(i > 0)
            {
            	//si j (presupuesto actual) es menor a el coste de la defensa actual
            	//entonces no puedo meter la defensa actual (me quedo igual que en la fila anterior)
                if(j < Candidatos[i].def->cost)
                    evaluacionTotal[i][j] = evaluacionTotal[i-1][j];
                    
                //si j (presupuesto actual) es mayor o igual a el coste de la defensa actual
                //entonces nos quedamos con el maximo de lo que tenia antes(fila anterior) y 
                //lo que tendria su meto en la mochila si meto la defensa actual en la mochila
                else
                    evaluacionTotal[i][j] = std::max(evaluacionTotal[i-1][j],
                    evaluacionTotal[i-1][j-Candidatos[i].def->cost] + Candidatos[i].valor);
            }
        }
    }
    //el maximo valor disponible para el conjunto de defensas estaria en la ultima posicion
    //tanto de fila y columna de la tabla de subproblemas resueltos 

	//variables para recorrer hacia atras los resultados de la mochila
    int j = ases, i = Candidatos.size() - 1;
    //lista de defensas seleccionadas (defensas que esten en la mochila)
    std::list<Defense*> DefensasSeleccionadas;
    //mientras j (el presupuesto) sea mayor que 0 y la fila sea mayor igual a 1
    while(j > 0 && i >= 1)
    {
    	//si la posicion actual es distinta a la superior (se ha metido la defensa)
        if(evaluacionTotal[i][j] != evaluacionTotal[i-1][j]) 
        {
        	//guardamos la defensa
            DefensasSeleccionadas.push_back(Candidatos[i].def);
            //al presupuesto se le quita lo que nos costo la defensa
            j = j - Candidatos[i].def->cost; 
        }   
        i--; //pasamos a la fila superior
    }

	//Ahora insertaremos las IDs de las defensas seleccionadas
    std::list<Defense*>::iterator dit = DefensasSeleccionadas.begin();
    while(dit != DefensasSeleccionadas.end())
    {
        selectedIDs.push_back((*dit)->id);
        dit++;
    }
}
\end{lstlisting}


\item Describa a continuación la función diseñada para otorgar un determinado valor a cada una de las celdas del terreno de batalla para el caso del resto de defensas. Suponga que el valor otorgado a una celda no puede verse afectado por la colocación de una de estas defensas en el campo de batalla. Dicho de otra forma, no es posible modificar el valor otorgado a una celda una vez que se haya colocado una de estas defensas. Evidentemente, el valor de una celda sí que puede verse afectado por la ubicación del centro de extracción de minerales.

En el caso de no realizar ordenacion, el algoritmo de selección tendría un coste de O(n²).

En el caso de la preordenación por fusion, el algoritmo de selección tendría un coste en el peor caso de O(nlogn).

En el caso de la preordenación rapida, el algoritmo de selección tendría un coste en el peor caso de O(n²).

En el caso de la preordenación por montículo, el algoritmo de selección tendría un coste de O(nlogn) (debido a la creacion del monticulo y luego a la operacion pop).


\item A partir de las funciones definidas en los ejercicios anteriores diseñe un algoritmo voraz que resuelva el problema global. Este algoritmo puede estar formado por uno o dos algoritmos voraces independientes, ejecutados uno a continuación del otro. Incluya a continuación el código fuente relevante que no haya incluido ya como respuesta al ejercicio 3. 

\begin{figure}[H]
	\includegraphics[width=1\textwidth]{Grafico.png}
\end{figure}


\end{enumerate}

Todo el material incluido en esta memoria y en los ficheros asociados es de mi autoría o ha sido facilitado por los profesores de la asignatura. Haciendo entrega de este documento confirmo que he leído la normativa de la asignatura, incluido el punto que respecta al uso de material no original.

\end{document}
