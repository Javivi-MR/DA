La función de factibilidad comprobará si una celda es factible para colocar la defensa o no. Para ello, tendremos que comprobar que no este fuera de rango, que no choque con un obstáculo o defensa ya colocada y que no salga del mapa. \\\\
Para que la casilla no este fuera de rango, no tiene que ser menor que 0 tanto en fila como en columna y no tiene que ser mayor o igual que los extremos del tablero.\\\\
Para que no choque con los obstáculos o defensas, la distancia entre el centro de la celda candidata y la celda del obstáculo/defensa ya colocada no debe ser menor que la suma de los radios de la defensa a colocar y el obstáculo/defensa ya colocada.\\\\
Para que no salga la defensa del mapa tenemos que comprobar que su radio no viola los limites del mapa. Para ello si la posición de la fila o la posición de la columna menos el radio de la defensa es menor que 0 sobrepasa alguno de los extremos inferiores. Mientras que si la posición de la fila o la posición de la columna mas el radio de la defensa es mayor que la anchura o la altura del mapa, entonces sobrepasa alguno de los extremos superiores.


\begin{lstlisting}[frame=single,basicstyle=\tiny,title={Función de factibilidad}]
bool EsFactible(int row, int col,List<Defense*>::iterator currentdefense
	, int nCellsWidth, int nCellsHeight, float mapWidth, float mapHeight, 
	List<Object*> obstacles, List<Defense*> defenses,int numDefensasColocadas)
{
	//definimos una variable factible, la predefinimos con true, 
	//y si encontramos que no es factible la pondremos a false
	bool factible = true; 
	
	//si la row y col de la celda pasada esta dentro de rango
	if(row >= 0 && row < nCellsWidth && col >= 0 && col < nCellsHeight) 
    {
		//definicon de variables necesarias para la verificacion de la factibilidad
		//obtenemos la posicion de la celda
		Vector3 pos = cellCenterToPosition(row,col,mapWidth/nCellsWidth
		,mapHeight/nCellsHeight);  
		//definimos un iterador para la lista de obstulos inicializado con la primera posicion
		List<Object*>::iterator poit = obstacles.begin(); 
		//definimos un iterador para la lista de defensas inicializado con la primera posicion
		List<Defense*>::iterator pdit = defenses.begin(); 

		//Comprobamos que los obstaculos no chocan con nuestra defensa
		//mientras que haya obstaculos por comprobar y factible sea verdadero
		while(poit != obstacles.end() && factible)	
		{
			//si la suma de los radios de la defensa a colocar y el radio del obstaculo es mayor que 
			//la distancia desde el centro de la posicion
			//de la celda hasta el centro de la posicion del obstaculo
			if(((*currentdefense)->radio + (*poit)->radio) > 
			_distance(pos,(*poit)->position)) 
				factible = false; //entonces la defensa no es colocable ya que estaria 
				//intersectando el radio del ocbstaculo (factible es falso)
			poit++; // pasamos al siguiente obstaculo
		}
		
		//Comprobamos que las defensas YA COLOCADAS no chocan con nuestra defensa
		//mientras que haya defensas por comprobar y factible sea verdadero
		while(pdit != defenses.end() && factible) 
		{
			//si la defensa que iteramos es la que vamos a colocar o 
			//numero de defensas colocadas es 0, no hay defensa que pueda colisionar
			if(pdit == currentdefense || numDefensasColocadas == 0) 
			{
				pdit++; //pasamos a la siguiente defensa
				continue; //terminamos la iteracion actual en el bucle
			}
			else
			{
				//si la suma del radios de la defensa a colocar 
				//y el radio de la defensa ya colocada es mayor que 
				//la distancia desde el centro de la posicion de la
				//celda hasta el centro de la posicion defensa
				if( ((*currentdefense)->radio + (*pdit)->radio) 
				> _distance(pos,(*pdit)->position))
					factible = false; //entonces la defensa no 
					//es colocable ya que estaria
				//intersectando el radio de la otra defensa (factible es falso)
				pdit++; // pasamos a la siguiente defensa
				numDefensasColocadas--; //hemos comprobado esta defensa, 
				//por lo que ya no la contamos
			}
		}

		//si la resta de la posicion x/y de la celda 
		//menos el radio de la defensa es menor que 0
		//o la suma de la posicion x/y de la celda 
		//mas el radio de la defensa es mayor que
		//la anchura o altura del mapa entonces el radio de la defensa
		//ocupa una parte inexistente del borde del mapa
		if(pos.x - (*currentdefense)->radio < 0 
		|| pos.y - (*currentdefense)->radio < 0 
		||pos.x + (*currentdefense)->radio > mapWidth 
		|| pos.y + (*currentdefense)->radio > mapHeight)
			factible = false; //entonces la defensa no es colocable ya que 
			//estaria violando los bordes del mapa (factible es falso)
	}
	else	//Si row y col esta fuera de rango
		factible = false; //entonces la defensa no es colocable (factible es falso)

	return factible; //devolvemos la factible
}

\end{lstlisting}
