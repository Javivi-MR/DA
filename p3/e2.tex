
\begin{lstlisting}[frame=single,basicstyle=\tiny,title={Algoritmo de Fusión}]
std::vector<Celda> fusion(std::vector<Celda>& FirstHalf,std::vector<Celda>& SecondHalf){

	std::vector<Celda> Results(FirstHalf.size() + SecondHalf.size());// Vector resultado (Vector ordenado creado por la fusion entre la primera mitad y la segunda)
	int i = 0, j = 0, k = 0;//Variables necesarias para recorrer cada vector
	
	
	
	while(i < FirstHalf.size() && j < SecondHalf.size()) //Mientras que no hayamos recorrido uno de los vectores enteros
	{
		if(FirstHalf[i] < SecondHalf[j]) //si la posicion i de la primera mitad es menor que la posicion j de la segunda mitad
		{
			Results[k] = FirstHalf[i];	//Asignamos el elemento en resultado
			i++;//iteramos i para la primera mitad
		}
		else
		{
			Results[k] = SecondHalf[j];//Asignamos el elemento en resultado
			j++; //iteramos j para la segunda mitad
		}
		k++; //iteramos k para el resultado
	}
	//Rellenar lo faltante de cada mitad
	while(i < FirstHalf.size())
	{
		Results[k] = FirstHalf[i];
		i++;
		k++;
	}
	while(j < SecondHalf.size())
	{
		Results[k] = SecondHalf[j];
		j++;
		k++;
	}
	return Results;
}


void ordenacionFusion(std::vector<Celda>& v)
{
    int n = v.size()/2; //obtenemos la posicion de la mitad del vector
    
    if(n > 0)	//si es mayor de 0
    {
    	std::vector<Celda> FirstHalf(n); //creamos 2 vectores auxiliares
    	std::vector<Celda> SecondHalf(v.size() - n);// Uno guardara la primera mitad y otro la segunda
    	
    	for(int i = 0; i < n; i++)	//relleno de la primera mitad
    	{
    		FirstHalf[i] = v[i];
    	}
    	
    	for(int i = n ; i < v.size(); i++) //relleno de la segunda mitad
    	{
    		SecondHalf[i-n] = v[i];
    	}
    	ordenacionFusion(FirstHalf); //llamada recursiva con la primera mitad
    	ordenacionFusion(SecondHalf);//llamada recursiva con la segunda mitad
    	v = fusion(FirstHalf,SecondHalf); //llamada a funsion
    }
}
\end{lstlisting}
