La estructura utilizada sera un vector que contendra cada una de las celdas del mapa. Ademas una celda contendra su posicion ademas de la valoración para dicha celda.

\begin{lstlisting}[frame=single,basicstyle=\tiny,title={Estructura de Celda}]
struct Celda
{
	int row,col;
	float valor;
	Celda(int r, int c, int v): row(r), col(c), valor(v) {}
	Celda(): row(0),col(0),valor(0) {}
};
\end{lstlisting}

\begin{lstlisting}[frame=single,basicstyle=\tiny,title={Vector de Celdas}]
	std::vector<Celda> Celdas;
	for(int i = 0 ; i < nCellsWidth ; i++)
	{
		for(int j = 0 ; j < nCellsHeight ; j++)
		{
			//insertamos todas las celdas en la lista de candidatos. Para poder ordenarla luego usaremos el metodo sort
			Celdas.push_back(Celda(i,j,defaultCellValue(i,j,freeCells,nCellsWidth,nCellsHeight,mapWidth,mapHeight,obstacles,defenses)));
		}
	}
\end{lstlisting}
