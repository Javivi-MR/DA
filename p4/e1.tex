El Algoritmo sigue la estructura de un algoritmo A*. En el tenemos, dos vectores (Nopen y Nclosed) que simbolizan las listas de abierto y cerrado, un nodo inicial y un nodo objetivo entre otras variables. Estas especialmente son esenciales para el funcionamiento del algoritmo. Lo primero que realiza es meter al nodo inicial dentro de la lista de abiertos y una vez abierto todos sus adyacentes (y después de haber calculado el valor de las funciones g, h, f y el valor adicional de celda para cada uno de ellos), mete a dicho nodo en la lista de cerrados, para posteriormente abrir el próximo nodo con menor valor en la función f. Esto es debido a que es el nodo que más cerca está de la solución. Para realizar las ordenaciones, usamos un montículo en el vector que representa la lista de abiertos. Además, gracias a la función estaEnVect, podemos hacer que no se añadan a la lista de abiertos nodos que ya estén contenidos en esta y nodos que estén contenidos en la lista de cerrados. Una vez llegamos al nodo objetivo, volvemos atrás gracias a la referencia al nodo padre de cada nodo, consiguiendo así el camino hacia el nodo objetivo y la introducción de este dentro de la lista de posiciones path.

También era necesario añadir la función comp para poder ordenar correctamente el montículo en función del valor de f para cada nodo.
