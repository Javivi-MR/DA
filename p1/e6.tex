\begin{lstlisting}[frame=single,basicstyle=\tiny,title={Voraz total}]
void DEF_LIB_EXPORTED placeDefenses(bool** freeCells, int nCellsWidth, 
			int nCellsHeight, float mapWidth, float mapHeight
            , std::list<Object*> obstacles, std::list<Defense*> defenses) {
	
    float cellWidth = mapWidth / nCellsWidth; //Ancho de celda
    float cellHeight = mapHeight / nCellsHeight; //Altura de celda
    int numDefensasColocadas = 0; //Numero de defensas colocadas

	//Definicion de variables que usaremos mas adelante:
	//Para la lista de candidatos usaremos una lista la 
	//cual almacenara las celdas de manera ordenada conforme al valor de la celda
	std::list<Celda> Celdas;
	//Definimos una celda auxiliar para poder seleccionar un candidato de la lista
	Celda auxiliar(0,0,0);
	//definimos un iterador de la lista de defensas para poder colocar las defensas
	List<Defense*>::iterator currentDefense = defenses.begin();
		
	//Inicializamos la lista de candidatos para colocar el centro de extraccion
	for(int i = 0 ; i < nCellsWidth ; i++)
	{
		for(int j = 0 ; j < nCellsHeight ; j++)
		{
			//insertamos todas las celdas en la lista de candidatos. Para poder 
			//ordenarla luego usaremos el metodo sort
			Celdas.push_front(Celda(i,j,cellValueExtrationCenter(i,j
			,freeCells,cellWidth,cellHeight,nCellsWidth,nCellsHeight)));
		}
	}
	Celdas.sort(); //Se realiza la ordenacion (al final de 
	//las listas estan las mejores celdas)

	//Comenzaremos colocando el centro de extraccion de minerales 
	//(primera posicion de defensas)
	//Definimos una variable que nos indicara true cuando 
	//el centro de extraccion sea colocado
	bool CentroColocado = false; 
	// Mientras que hayan candidatos(celdas)disponibles 
	//y el centro no haya sido colocado
	while(!Celdas.empty() && !CentroColocado) 
	{
		auxiliar = Celdas.back(); //Guardamos el mejor candidato posible
		Celdas.pop_back(); //lo eliminamos de la lista de candidatos
		//Si la celda elegida es factible
		if(EsFactible(auxiliar.row,auxiliar.col,currentDefense,nCellsWidth,
		nCellsHeight,mapWidth,mapHeight,
		obstacles,defenses,numDefensasColocadas))  
		{
			//le damos a la primera defensa (Centro de Extraccion) 
			//la posicion de la celda elegida
			(*currentDefense)->position = 
			cellCenterToPosition(auxiliar.row,auxiliar.col,
			cellWidth,cellHeight); 
			//El centro ha sido colocado
			CentroColocado = true; 
			//numero de defensas es 1
			numDefensasColocadas++; 
		}
	}
	
	//Actualizamos el valor de las celdas para que sea el mejor valor para las celdas
	cellValueDefenses(Celdas,currentDefense,cellWidth,cellHeight); 
	//Se realiza la ordenacion (al final de las listas estan las mejores celdas
	Celdas.sort();
	
	//pasamos a la siguiente defensa
	currentDefense++; 

	//Continuamos con el colocado de las defensas restantes
	//establecemos un numero maximo de intentos
	int maxAttemps = 1000;
	//Mientras que haya defensas por colocar y haya candidatos disponibles 
	//y el numero de intentos sea superior a 0
    while(currentDefense != defenses.end() && !Celdas.empty() && maxAttemps > 0) 
    {	 
		//seleccionamos el mejor candidato disponible
		auxiliar = Celdas.back(); 
		//lo eliminamos de la lista de candidatos
		Celdas.pop_back();
		//Si la celda elegida es factible
    	if(EsFactible(auxiliar.row,auxiliar.col,currentDefense,nCellsWidth,
	nCellsHeight,mapWidth,mapHeight,obstacles,defenses,numDefensasColocadas)) 
    	{
			//asignamos a la defensa la posicion de la celda elegida
        	(*currentDefense)->position = 
		cellCenterToPosition(auxiliar.row,
		auxiliar.col,cellWidth,cellHeight); 
			//pasamos a la siguiente defensa
			currentDefense++; 
			//defensa colocada +1
			numDefensasColocadas++; 
    	}
		//restamos 1 al numero de intentos
		maxAttemps--; 
    }

	
\end{lstlisting}
